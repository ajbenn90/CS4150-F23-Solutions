\documentclass{article}
\usepackage{graphicx} % Required for inserting images
%\usepackage{geometry}
%\geometry{margin=1in}
\usepackage[margin=1in]{geometry}
\usepackage{algpseudocode}
\usepackage{algorithm}
\usepackage{url}
\usepackage{amsmath, amsthm, amssymb}
\usepackage{booktabs}
\usepackage{pgfplotstable}
\usepackage{tikz-cd}
\usepackage{tikz}
\usetikzlibrary{arrows}
\usepackage{hyperref}
\usepackage{float}


\begin{document}
\subsection*{Problem 1}
Pick the gas station that is farthest away from you that you can still reach given you gas tank.

\subsection*{Problem 2}
Swap the two stations $o_1$ and $g_1$. Since the greedy picks the farthest station away, then either $o_1$ \textit{is} $g_1$ or $o_1$ is closer than $g_1$, so we can still make it there. 

\subsection*{Problem 3}
We can continually swap $o_i$ for $g_i$ until we run out of gas stations. The greedy must have the same number of stops, otherwise the optimal would not have been optimal.\\

\noindent Note that the greedy cannot have more stops than the optimal since we always choose the farthest away gas station.

\subsection*{Problem 4}
No, counter example: $S = 4$, pairs: $(4,3), (2,2), (1,2)$. Greedy chooses $(4,3)$, but optimally we would choose $(2,2), (1,2)$.

\subsection*{Problem 5}
No, counter example: $S = 2$, pairs: $(1,4), (2,7)$. Greedy chooses $(1,4)$ but optimally we want $(2,7)$.

\end{document}
 