\documentclass{article}
\usepackage{graphicx} % Required for inserting images
%\usepackage{geometry}
%\geometry{margin=1in}
\usepackage[margin=1in]{geometry}
\usepackage{algpseudocode}
\usepackage{algorithm}
\usepackage{url}
\usepackage{amsmath, amsthm, amssymb}
\usepackage{booktabs}
\usepackage{pgfplotstable}
\usepackage{tikz-cd}
\usepackage{tikz}
\usetikzlibrary{arrows}
\usepackage{hyperref}
\usepackage{float}


\begin{document}
\subsection*{Problem 1}
Answer 1: Eating nothing gives 0 satisfaction, which is always the smallest\\
Answer 2: Starting at 0 causes an infinite loop

\subsection*{Problem 2}
At most $n-1$ calls

\subsection*{Problem 4}
$n^2$ subproblems, $n$ work in each. Total $O(n^3)$

\subsection*{Problem 5}
Max of day $j+1$ is equal to the max we could eat on day $j$. We also need to multiply by the decay factor too since we are eating it on day $j+1$. Hence $MaxS(n, j+1) = \beta \cdot MaxS(n,j)$

\subsection*{Problem 6}
$\beta \cdot \sqrt{k} + maxS(n-k)$\\

\noindent Optional: $O(n^2)$ complexity, $O(n)$ space (only $n$ distinct problems).
\end{document}
 